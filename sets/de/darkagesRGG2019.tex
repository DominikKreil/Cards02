% Basic settings for this card set
\renewcommand{\cardcolor}{darkages}
\renewcommand{\cardextension}{Erweiterung VI}
\renewcommand{\cardextensiontitle}{Dark Ages}
\renewcommand{\seticon}{darkages.png}

\clearpage
\newpage
\section{\cardextension \ - \cardextensiontitle \ (Rio Grande Games 2019)}

\begin{tikzpicture}
	\card
	\cardstrip
	\cardbanner{banner/white.png}
	\cardicon{icons/coin.png}
	\cardprice{1}
	\cardtitle{Armenhaus}
	\cardcontent{Befindet sich unter den aufgedeckten Handkarten keine Geldkarte, bleibt es bei +\coin[4]. Hast du 1 Geldkarte auf der Hand, reduziert sich dies auf +\coin[3] usw. Hast du weitere Geldwerte im Spiel (z.B. durch andere Königreichkarten) kannst du, wenn du mehr als \coin[4] Geldkarten auf deiner Hand hast, auch mehr als 4 verlieren, gehst dabei aber nie insgesamt unter \coin[0].}
\end{tikzpicture}
\hspace{-0.6cm}
\begin{tikzpicture}
	\card
	\cardstrip
	\cardbanner{banner/blue.png}
	\cardicon{icons/coin.png}
	\cardprice{2}
	\cardtitle{Bettler}
	\cardcontent{Wenn du diese Karte ausspielst, nimm drei \emph{KUPFER} vom Vorrat direkt auf die Hand. Sind nicht genügend \emph{KUPFER} vorhanden, nimm so viele, wie du kannst.

	\smallskip

	Wenn ein Mitspieler eine Angriffskarte ausspielt und du diese Karte zu diesem Zeitpunkt auf der Hand hast, darfst du sie ablegen. Wenn du das tust, nimmst du dir 2 \emph{SILBER}. Lege eines davon ab und das andere auf deinen Nachziehstapel. Kannst du nur 1 \emph{SILBER} nehmen, weil nicht genügend \emph{SILBER} im Vorrat sind, lege dieses auf deinen Nachziehstapel.}
\end{tikzpicture}
\hspace{-0.6cm}
\begin{tikzpicture}
	\card
	\cardstrip
	\cardbanner{banner/white.png}
	\cardicon{icons/coin.png}
	\cardprice{2}
	\cardtitle{Knappe}
	\cardcontent{Du erhältst auf jeden Fall +\coin[1]. Dann wähle, ob du + 2 Aktionen oder + 2 Käufe
	oder ein \emph{SILBER} vom Vorrat erhalten möchtest.

	\smallskip

	Wenn du diese Karte entsorgst, nimm eine Angriffskarte vom Vorrat. Ist keine solche Karte vorhanden, erhältst du nichts. Diese Karte selbst gibt dir nicht das Recht, sich selbst zu entsorgen. Du kannst sie nur entsorgen, wenn eine andere Karte dies anweist.}
\end{tikzpicture}
\hspace{-0.6cm}
\begin{tikzpicture}
	\card
	\cardstrip
	\cardbanner{banner/white.png}
	\cardicon{icons/coin.png}
	\cardprice{2}
	\cardtitle{\footnotesize{Landstreicher}}
	\cardcontent{Ist die aufgedeckte Karte ein \emph{FLUCH}, eine RUINE, ein UNTERSCHLUPF oder eine Punktekarte, nimm sie auf deine Hand. Ansonsten lege sie zurück auf deinen Nachziehstapel.}
\end{tikzpicture}
\hspace{-0.6cm}
\begin{tikzpicture}
	\card
	\cardstrip
	\cardbanner{banner/white.png}
	\cardicon{icons/coin.png}
	\cardprice{3}
	\cardtitle{Eremit}
	\cardcontent{Wenn du diese Karte ausspielst, darfst du eine Karte entsorgen, die nicht den Typ GELD enthält – diese darf entweder aus deinem Ablagestapel oder aus deiner Hand stammen. Egal ob du eine Karte entsorgt hast oder nicht, nimm eine Karte vom Vorrat, die bis zu \coin[3] kostet. Karten mit \hex (aus \emph{Empires}) bzw. \potion (aus \emph{Alchemisten}) in den Kosten sowie Karten, die nicht zum Vorrat gehören (z.B. \emph{VERRÜCKTER}), dürfen nicht genommen werden.

	\smallskip

	Wenn du diese Karte im Spiel hast und dann ablegst und du in deinem Zug bis zu diesem Zeitpunkt keine Karte gekauft hast (du darfst aber Karten auf andere Art und Weise genommen haben), entsorge diesen \emph{EREMITEN} und nimm einen \emph{VERRÜCKTEN} von seinem Stapel. Wird der \emph{EREMIT} nicht abgelegt, z.B. weil er mit der \emph{PROZESSION} zweimal gespielt und dadurch schon aus dem Spiel entsorgt wurde, nimmst du dir keinen \emph{VERRÜCKTEN}.

	\smallskip

	Spielvorbereitung: Legt den \emph{VERRÜCKTEN-Stapel} neben den Vorrat bereit. }
\end{tikzpicture}
\hspace{-0.6cm}
\begin{tikzpicture}
	\card
	\cardstrip
	\cardbanner{banner/white.png}
	\cardicon{icons/coin.png}
	\cardprice{3}
	\cardtitle{Gassenjunge}
	\cardcontent{Mitspieler, die bereits 4 oder weniger Karten auf ihrer Hand haben,
	müssen keine Karten ablegen.
	Wenn du einen \emph{GASSENJUNGEN} im Spiel hast und du eine Angriffskarte ausspielst (auch einen anderen \emph{GASSENJUNGEN}), darfst du, bevor du diese Angriffskarte ausführst, diesen \emph{GASSENJUNGEN} entsorgen. Wenn du das tust, nimm dir einen \emph{SÖLDNER} vom \emph{SÖLDNER}-Stapel. Spielst du einen \emph{GASSENJUNGEN} mit einer \emph{PROZESSION}, kannst du den \emph{GASSENJUNGEN} nicht entsorgen, um einen \emph{SÖLDNER} zu nehmen, sondern du musst ihn auf Grund der Anweisung der \emph{PROZESSION} entsorgen, um eine Karte zu nehmen, die genau \coin[1] mehr als der entsorgte \emph{GASSENJUNGE} kostet.}
\end{tikzpicture}
\hspace{-0.6cm}
\begin{tikzpicture}
	\card
	\cardstrip
	\cardbanner{banner/white.png}
	\cardicon{icons/coin.png}
	\cardprice{3}
	\cardtitle{Lagerraum}
	\cardcontent{Wenn du zum ersten Mal Handkarten ablegst, ziehe pro abgelegter Karte eine Karte nach. Lege dann wieder beliebig viele deiner Handkarten ab (darunter können auch Karten sein, die du gerade gezogen hast; du kannst auch gar keine Karten ablegen) und erhalte +\coin[1] pro nun abgelegter Karte.}
\end{tikzpicture}
\hspace{-0.6cm}
\begin{tikzpicture}
	\card
	\cardstrip
	\cardbanner{banner/blue.png}
	\cardicon{icons/coin.png}
	\cardprice{3}
	\cardtitle{Marktplatz}
	\cardcontent{Wenn eine deiner Karten entsorgt wird (egal ob in deinem Zug oder dem Zug eines Mitspielers, egal ob du die Karte entsorgen willst oder nicht) und du zu diesem Zeitpunkt einen \emph{MARKTPLATZ} auf deiner Hand hast, darfst du diesen ablegen. Wenn du das tust, nimm ein \emph{GOLD} vom Vorrat. Hast du mehrere \emph{MARKTPLÄTZE} auf deiner Hand, wenn du eine Karte entsorgst, darfst du beliebig viele \emph{MARKTPLÄTZE} ablegen und entsprechend viele \emph{GOLD} nehmen.}
\end{tikzpicture}
\hspace{-0.6cm}
\begin{tikzpicture}
	\card
	\cardstrip
	\cardbanner{banner/white.png}
	\cardicon{icons/coin.png}
	\cardprice{3}
	\cardtitle{Mundraub}
	\cardcontent{Wenn du mindestens eine Karte auf deiner Hand hast, musst du eine deiner Handkarten entsorgen. Auch wenn du das nicht tun kannst, schaust du den Müllstapel durch. Pro Geldkarte mit unterschiedlichem Namen, die sich im Müll befindet, erhältst du +\coin[1]. Befinden sich dort z.B. 3 \emph{KUPFER}, 1 \emph{SILBER} und 2 \emph{FALSCHGELD}, erhältst du +\coin[3].}
\end{tikzpicture}
\hspace{-0.6cm}
\begin{tikzpicture}
	\card
	\cardstrip
	\cardbanner{banner/white.png}
	\cardicon{icons/coin.png}
	\cardprice{3}
	\cardtitle{Weiser}
	\cardcontent{Wenn du (auch nach dem Mischen deines Ablagestapels) keine Karte findest, die \coin[3] oder mehr kostet, nimm keine Karte auf deine Hand. Lege alle aufgedeckten Karten ab.}
\end{tikzpicture}
\hspace{-0.6cm}
\begin{tikzpicture}
	\card
	\cardstrip
	\cardbanner{banner/white.png}
	\cardicon{icons/coin.png}
	\cardprice{4}
	\cardtitle{Barde}
	\cardcontent{Hast du nicht genügend Karten in deinem Nachziehstapel, um 3 Karten aufzudecken, mische deinen Ablagestapel, lege die gemischten Karten unter deinen Nachziehstapel und decke dann 3 Karten auf. Deckst du dabei keine Aktionskarten auf, legst du alle Karten ab und keine Karten zurück auf deinen Nachziehstapel.}
\end{tikzpicture}
\hspace{-0.6cm}
\begin{tikzpicture}
	\card
	\cardstrip
	\cardbanner{banner/white.png}
	\cardicon{icons/coin.png}
	\cardprice{4}
	\cardtitle{\footnotesize{Eisenhändler}}
	\cardcontent{Je nach Kartentyp (\emph{AKTION} – \emph{GELD} – \emph{PUNKTE} \dots) der aufgedeckten Karte erhältst du einen Bonus. Für Karten mit mehreren Kartentypen erhältst du jeden der entsprechenden Boni. Du darfst wählen, ob du die aufgedeckte Karte zurück auf deinen Nachziehstapel legst oder ablegst.}
\end{tikzpicture}
\hspace{-0.6cm}
\begin{tikzpicture}
	\card
	\cardstrip
	\cardbanner{banner/white.png}
	\cardicon{icons/coin.png}
	\cardprice{4}
	\cardtitle{Festung}
	\cardcontent{Wenn du diese Karte entsorgst (egal ob du das musst oder willst), entsorge sie in den Müll und nimm sie dann zurück auf deine Hand. Eine so zurückgenommene \emph{FESTUNG} gilt im regeltechnischen Sinn nicht als \enquote{genommene Karte}.}
\end{tikzpicture}
\hspace{-0.6cm}
\begin{tikzpicture}
	\card
	\cardstrip
	\cardbanner{banner/green.png}
	\cardicon{icons/coin.png}
	\cardprice{4}
	\cardtitle{Lehen}
	\cardcontent{Als Punktekarte ist das \emph{LEHEN} pro 3 \emph{SILBER} im eigenen Kartensatz 1 \victorypoint wert (abgerundet). Hast du zum Beispiel 11 \emph{SILBER} im Kartensatz, ist jedes \emph{LEHEN} 3 \victorypoint wert.

	\smallskip

	Wenn du das \emph{LEHEN} entsorgst, nimm 3 \emph{SILBER} vom Vorrat. Diese Karte selbst gibt dir nicht das Recht, sich selbst zu entsorgen. Du kannst sie nur entsorgen, wenn eine andere Karte dies anweist, erhältst dafür aber zu Spielende keine Siegpunkte mit Hilfe dieses \emph{LEHENS}.

	\smallskip

	Spielvorbereitung: Legt entsprechend der Spieleranzahl folgende Anzahl \emph{LEHEN} in den Vorrat: bei 2 Spielern: 8 Karten und bei 3+ Spielern: 12 Karten.}
\end{tikzpicture}
\hspace{-0.6cm}
\begin{tikzpicture}
	\card
	\cardstrip
	\cardbanner{banner/white.png}
	\cardicon{icons/coin.png}
	\cardprice{4}
	\cardtitle{\footnotesize{Leichenkarren}}
	\cardcontent{Wenn du diese Karte ausspielst, erhältst du +\coin[5] und musst entweder eine beliebige Aktionskarte aus deiner Hand oder diesen \emph{LEICHENKARREN} entsorgen. Hast du keine Aktionskarte(n) auf deiner Hand, musst du den \emph{LEICHENKARREN} entsorgen.

	\smallskip

	Wenn du diese Karte nimmst (egal ob du sie kaufst oder auf andere Art und Weise nimmst, ggf. während des Zuges eines Mitspielers), musst du die beiden obersten RUINEN vom RUINEN-Stapel nehmen; zeige sie deinen Mitspielern. Du nimmst die zwei RUINEN aber nur, wenn du wirklich einen \emph{LEICHENKARREN} nimmst – nimmst du stattdessen z.B. mit Hilfe des \emph{FAHRENDEN HÄNDLERS} (aus \emph{Hinterland}) ein \emph{SILBER}, nimmst du die RUINEN nicht.}
\end{tikzpicture}
\hspace{-0.6cm}
\begin{tikzpicture}
	\card
	\cardstrip
	\cardbanner{banner/white.png}
	\cardicon{icons/coin.png}
	\cardprice{4}
	\cardtitle{\scriptsize{Lumpensammler}}
	\cardcontent{Deinen kompletten Nachziehstapel auf den Ablagestapel zu legen ist optional. Die so auf den Ablagestapel gelegten Karten gelten nicht als \enquote{abgelegt}. Wenn du so z.B. einen \emph{TUNNEL} (aus \emph{Hinterland}) auf den Ablagestapel legst, kommt dessen Reaktion auf das Ablegen des \emph{TUNNELS} außerhalb der Aufräumphase nicht zum Tragen. Egal ob du deinen Nachziehstapel auf den Ablagestapel gelegt hast oder nicht, musst du deinen Ablagestapel durchsehen und irgendeine Karte daraus auf deinen Nachziehstapel legen (außer du hast keine Karten in deinem Ablagestapel).}
\end{tikzpicture}
\hspace{-0.6cm}
\begin{tikzpicture}
	\card
	\cardstrip
	\cardbanner{banner/white.png}
	\cardicon{icons/coin.png}
	\cardprice{4}
	\cardtitle{Marodeur}
	\cardcontent{Nimm zuerst eine \emph{BEUTE} vom \emph{BEUTE}-Stapel – dieser ist nicht Teil des Vorrats. Ist der \emph{BEUTE}-Stapel leer, erhältst du keine. Dann nimmt jeder Mitspieler – beginnend bei deinem linken Nachbarn – die oberste RUINE vom RUINEN-Stapel. Wird dieser dabei leer, erhalten die verbleibenden Mitspieler keine RUINE.}
\end{tikzpicture}
\hspace{-0.6cm}
\begin{tikzpicture}
	\card
	\cardstrip
	\cardbanner{banner/white.png}
	\cardicon{icons/coin.png}
	\cardprice{4}
	\cardtitle{Ratten}
	\cardcontent{Du musst eine deiner Handkarten, außer \emph{RATTEN}, entsorgen, wenn du eine auf der Hand hast. Decke deine Handkarten auf, wenn dies nicht der Fall ist.

	\smallskip

	Wenn du eine \emph{RATTEN} entsorgst (egal ob in deinem Zug oder im Zug eines Mitspielers), ziehe eine Karte nach. Diese Karte selbst gibt dir nicht das Recht, sich selbst zu entsorgen.

	\smallskip

	Du kannst sie nur entsorgen, wenn das durch einen anderen Effekt im Spiel (z.B. die Anweisung auf einer anderen Karte) ausgelöst wird. Anders als die meisten anderen Königreichkarten ist \emph{RATTEN} 20x im Spiel enthalten.

	\smallskip

	Spielvorbereitung: Legt immer alle 20 \emph{RATTEN} im Vorrat bereit.}
\end{tikzpicture}
\hspace{-0.6cm}
\begin{tikzpicture}
	\card
	\cardstrip
	\cardbanner{banner/white.png}
	\cardicon{icons/coin.png}
	\cardprice{4}
	\cardtitle{Prozession}
	\cardcontent{\tiny{\begin{Spacing}{1}
	\vspace{1em}
	Eine Aktionskarte aus deiner Hand auszuspielen ist optional. Wenn du es tust, spiele sie aus, führe ihre Anweisungen aus, spiele sie ein zweites Mal aus (ohne sie zwischendurch auf die Hand zurückzunehmen) und entsorge die Aktionskarte dann. Du musst dann eine Aktionskarte vom Vorrat nehmen, die genau \coin[1] mehr kostet als die entsorgte Karte (außer es gibt keine Karte, die genau \coin[1] mehr kostet). Sobald du mit dem Ausspielen der auf die \emph{PROZESSION} folgenden Aktionskarte begonnen hast, darfst du keine anderen Karten dazwischen spielen (außer die ausgespielte Aktionskarte selbst weist dies an). Passiert etwas beim Entsorgen der Karte, wird dies ausgeführt, bevor du eine Karte nimmst.

	\smallskip

	Spielst du zum Beispiel mit einer \emph{PROZESSION} eine zweite \emph{PROZESSION} aus, darfst du diese zweite \emph{PROZESSION} dank der ersten doppelt ausspielen und dadurch insgesamt 2 Aktionskarten aus deiner Hand doppelt ausspielen. Du musst sowohl die zwei Aktionskarten aus deiner Hand als auch die zweite \emph{PROZESSION} nach dem Abhandeln entsorgen und dir jeweils eine Aktionskarte nehmen, die genau \coin[1] mehr kostet.

	\smallskip

	Spielst du mit einer \emph{PROZESSION} eine Dauerkarte (z.B. aus \emph{Seaside}, \emph{Abenteuer} oder \emph{Empires}), musst du die Dauerkarte entsorgen. In der Aufräumphase dieses Zuges legst du die \emph{PROZESSION} ab. Der Effekt der Dauerkarte wirkt trotzdem so lange, wie die Dauerkarte sonst im Spiel geblieben wäre.
	\end{Spacing}}}
\end{tikzpicture}
\hspace{-0.6cm}
\begin{tikzpicture}
	\card
	\cardstrip
	\cardbanner{banner/white.png}
	\cardicon{icons/coin.png}
	\cardprice{4}
	\cardtitle{\footnotesize{Waffenkammer}}
	\cardcontent{Karten mit \hex (aus \emph{Empires}) bzw. \potion (aus \emph{Alchemisten}) in den Kosten sowie Karten, die nicht zum Vorrat gehören (z.B. \emph{VERRÜCKTER}), dürfen nicht genommen werden. Lege die genommene Karte auf deinen Nachziehstapel.}
\end{tikzpicture}
\hspace{-0.6cm}
\begin{tikzpicture}
	\card
	\cardstrip
	\cardbanner{banner/white.png}
	\cardicon{icons/coin.png}
	\cardprice{5}
	\cardtitle{\footnotesize{Banditenlager}}
	\cardcontent{Ziehe eine Karte von deinem Nachziehstapel, bevor du eine \emph{BEUTE} vom \emph{BEUTE}-Stapel – dieser ist nicht Teil des Vorrats – nimmst. Ist keine \emph{BEUTE} mehr auf dem Stapel, erhältst du keine.}
\end{tikzpicture}
\hspace{-0.6cm}
\begin{tikzpicture}
	\card
	\cardstrip
	\cardbanner{banner/gold.png}
	\cardicon{icons/coin.png}
	\cardprice{5}
	\cardtitle{Falschgeld}
	\cardcontent{Diese Karte ist eine Geldkarte mit zusätzlichen Anweisungen. Sie ist \coin[1] wert und du erhältst + 1 Kauf. Wenn du ein \emph{FALSCHGELD} ausspielst und du eine oder mehrere Geldkarten auf der Hand hast (das darf auch ein weiteres \emph{FALSCHGELD} sein), darfst du eine davon ausspielen, das entsprechende Geld erhalten (und ggf. zusätzliche Anweisungen jener Geldkarte befolgen) und sie dann sofort noch ein weiteres Mal ausspielen, das Geld ein zweites Mal erhalten (und auch die ggf. zusätzliche Anweisung ein zweites Mal befolgen). Wenn du das tust, musst du die doppelt ausgespielte Geldkarte nach der 2. Ausführung entsorgen.}
\end{tikzpicture}
\hspace{-0.6cm}
\begin{tikzpicture}
	\card
	\cardstrip
	\cardbanner{banner/white.png}
	\cardicon{icons/coin.png}
	\cardprice{5}
	\cardtitle{Grabräuber}
	\cardcontent{Du kannst den Müllstapel durchschauen, bevor du dich für eine der beiden Optionen entscheidest. Du kannst dich auch für eine Option entscheiden, die du nicht oder nur teilweise erfüllen kannst. Nimmst du eine Karte vom Müll, die \coin[3] bis \coin[6] kostet, zeige sie deinen Mitspielern und lege sie auf deinen Nachziehstapel. Karten mit \hex (aus \emph{Empires}) bzw. \potion (aus \emph{Alchemisten}) dürfen nicht genommen werden.

	\smallskip

	Entscheidest du dich, eine Aktionskarte aus deiner Hand zu entsorgen, muss die genommene Karte vom Vorrat (nicht von einem Nicht-Vorratsstapel) stammen. Lege die genommene Karte auf deinen Ablagestapel.}
\end{tikzpicture}
\hspace{-0.6cm}
\begin{tikzpicture}
	\card
	\cardstrip
	\cardbanner{banner/white.png}
	\cardicon{icons/coin.png}
	\cardprice{5}
	\cardtitle{Graf}
	\cardcontent{Diese Karte gibt dir zwei aufeinanderfolgende Wahlmöglichkeiten mit jeweils 3 Optionen: Zuerst entscheidest du dich, entweder 2 Handkarten abzulegen oder eine Handkarte auf den Nachziehstapel zu legen oder ein K\emph{UPFER} vom Vorrat zu nehmen. Wenn du das getan hast, wählst du, entweder +\coin[3] zu nehmen, alle Handkarten zu entsorgen oder dir ein \emph{HERZOGTUM} vom Vorrat zu nehmen. Du kannst eine Option auch wählen, wenn du sie gar nicht erfüllen kannst.}
\end{tikzpicture}
\hspace{-0.6cm}
\begin{tikzpicture}
	\card
	\cardstrip
	\cardbanner{banner/white.png}
	\cardicon{icons/coin.png}
	\cardprice{5}
	\cardtitle{Katakomben}
	\cardcontent{Wenn du diese Karte ausspielst, siehst du dir die obersten drei Karten deines Nachziehstapels an, zeigst sie deinen Mitspielern aber nicht. Nimm entweder alle drei auf deine Hand oder lege alle drei Karten ab. Du ziehst nur 3 Karten, wenn du die 3 Karten ablegst. Musst du deinen Ablagestapel mischen, um die drei Karten nachzuziehen, lege erst die Karten ab und mische sie zusammen mit dem restlichen Ablagestapel.

	\smallskip

	Wenn du diese Karte entsorgst, nimmst du eine Karte, die weniger kostet als diese \emph{KATAKOMBEN}, auch wenn diese Karte im Zug eines anderen Spielers oder durch die Karte eines anderen Spielers entsorgt wird. Diese Karte selbst gibt dir nicht das Recht, sich selbst zu entsorgen. Du kannst sie nur entsorgen, wenn das durch einen anderen Effekt im Spiel (z.B. die Anweisung auf einer anderen Karte) ausgelöst wird.}
\end{tikzpicture}
\hspace{-0.6cm}
\begin{tikzpicture}
	\card
	\cardstrip
	\cardbanner{banner/white.png}
	\cardicon{icons/coin.png}
	\cardprice{5}
	\cardtitle{Kultist}
	\cardcontent{Beginnend bei deinem linken Mitspieler, nimmt jeder Mitspieler die oberste RUINE vom RUINEN-Stapel und legt sie ab. Wird der Stapel dabei leer, erhalten die nachfolgenden Mitspieler keine RUINE mehr. Danach darfst du einen weiteren \emph{KULTISTEN} aus deiner Hand ausspielen.

	\smallskip

	Wenn du diese Karte entsorgst, ziehst du drei Karten von deinem Nachziehstapel nach. Diese Karte selbst gibt dir nicht das Recht, sich selbst zu entsorgen. Du kannst sie nur entsorgen, wenn das durch einen anderen Effekt im Spiel (z.B. die Anweisung auf einer anderen Karte) ausgelöst wird.}
\end{tikzpicture}
\hspace{-0.6cm}
\begin{tikzpicture}
	\card
	\cardstrip
	\cardbanner{banner/white.png}
	\cardicon{icons/coin.png}
	\cardprice{5}
	\cardtitle{Medium}
	\cardcontent{Nenne den Namen einer Karte – z.B. \emph{GOLD} oder \emph{GRABRÄUBER}, nicht Aktionskarte oder Geld – dies sind Kartentypen, keine Namen. Karten von gemischten Stapeln, wie z.B. dem RITTER-Stapel, müssen konkret genannt werden, z.B. \emph{DAME NATALIE}, nicht einfach RITTER. Es ist kein \emph{MUSS} eine Karte zu nennen, die in diesem Spiel verwendet wird. Decke die oberste Karte deines Nachziehstapels auf. Ist es die genannte Karte, nimm sie auf deine Hand. Ansonsten lege sie zurück.}
\end{tikzpicture}
\hspace{-0.6cm}
\begin{tikzpicture}
	\card
	\cardstrip
	\cardbanner{banner/white.png}
	\cardicon{icons/coin.png}
	\cardprice{5}
	\cardtitle{Neubau}
	\cardcontent{Du kannst jede Karte nennen, die du willst, egal ob es sich um eine Punktekarte oder nicht handelt, egal ob sie in diesem Spiel verwendet wird oder nicht. Hast du eine Nicht-Punktekarte genannt, deckst du so lange Karten auf, bis du die erste Punktekarte (auch ggf. eine kombinierte) aufdeckst. Hast du eine Punktekarte genannt, deckst du so lange Karten auf, bis du die erste Punktekarte aufdeckst, die du nicht genannt hast. Diese musst du entsorgen und dir eine Punktekarte aus dem Vorrat nehmen, die bis zu \coin[3] mehr kostet. Das kann auch die gleiche wie die entsorgte Punktekarte oder eine billigere sein.}
\end{tikzpicture}
\hspace{-0.6cm}
\begin{tikzpicture}
	\card
	\cardstrip
	\cardbanner{banner/white.png}
	\cardicon{icons/coin.png}
	\cardprice{5}
	\cardtitle{Raubzug}
	\cardcontent{Entsorge zuerst diese Karte, dann decken deine Mitspieler mit 5 oder mehr Handkarten ihre Karten auf, von denen du jeweils bestimmst, welche sie ablegen müssen. Schließlich nimmst du dir 2 \emph{BEUTEN} vom \emph{BEUTE}-Stapel. Sind nicht mehr genügend vorhanden, nimm so viele, wie du kannst.}
\end{tikzpicture}
\hspace{-0.6cm}
\begin{tikzpicture}
	\card
	\cardstrip
	\cardbanner{banner/white.png}
	\cardicon{icons/coin.png}
	\cardprice{5}
	\cardtitle{Ritter}
	\cardcontent{\tiny{\begin{Spacing}{1}
	\vspace{2em}
	Dieser Kartenstapel ist ein gemischter Stapel. Jede Karte hat einen eigenen Namen und ist im Stapel genau 1x vorhanden. Mischt in der Spielvorbereitung alle RITTER, legt den RITTER-Stapel verdeckt in den Vorrat und deckt nur die oberste Karte auf. Immer die oberste Karte des RITTER-Stapels kann genommen werden.

	\smallskip

	Jede Karte enthält eine gleiche Anweisung, die einen Angriff darstellt: Jeder Mitspieler muss die obersten 2 Karten seines Nachziehstapels aufdecken. Kostet mindestens eine davon \coin[3] bis \coin[6], muss er diese entsorgen. Karten mit \hex (aus \emph{Empires}) bzw. \potion (aus \emph{Alchemisten}) in den Kosten kosten nie \coin[3] bis \coin[6]. Wenn beide \coin[3] bis \coin[6] kosten, darf er wählen, welche er entsorgt. Entsorgt mindestens ein Spieler auf diese Weise einen RITTER, musst du den von dir ausgespielten RITTER entsorgen.

	\smallskip

	Darüber hinaus enthält jeder RITTER eine zusätzliche Anweisung, die auf jedem RITTER unterschiedlich ist – hierbei gelten folgende zusätzliche Hinweise:
	\begin{itemize}
		\item \emph{DAME JOSEPHINE} ist auch eine Punktekarte – macht aber, auch wenn sie zu Beginn des Spiels oben auf dem Stapel liegt, nicht den gesamten Stapel zum Punktestapel. So kann die \emph{HANDELSROUTE} (aus \emph{Blütezeit}) keinen Marker auf den RITTER-Stapel legen.
		\item \emph{DAME ANNA}: Du darfst 0, 1 oder 2 deiner Handkarten entsorgen.
		\item \emph{DAME NATALIE}: Die genommene Karte muss aus dem Vorrat stammen.
		\item \emph{SIR VANDER}: Es ist egal, ob du diesen RITTER auf Grund der Anweisung auf \emph{SIR VANDER} oder auf Grund einer Anweisung einer anderen Karte entsorgst (auch außerhalb deines Zuges) – nimm ein \emph{GOLD} vom Vorrat.
		\item \emph{SIR MICHAEL}: Die Mitspieler müssen zuerst Karten ablegen, bis sie nur noch 3 Handkarten haben und dann Karten aufdecken und ggf. entsorgen.
	\end{itemize}
	\end{Spacing}}}
\end{tikzpicture}
\hspace{-0.6cm}
\begin{tikzpicture}
	\card
	\cardstrip
	\cardbanner{banner/white.png}
	\cardicon{icons/coin.png}
	\cardprice{5}
	\cardtitle{\scriptsize{Schrotthändler}}
	\cardcontent{Wenn du mindestens eine Karte auf deiner Hand hast, musst du eine Handkarte entsorgen.}
\end{tikzpicture}
\hspace{-0.6cm}
\begin{tikzpicture}
	\card
	\cardstrip
	\cardbanner{banner/white.png}
	\cardicon{icons/coin.png}
	\cardprice{5}
	\cardtitle{Schurke}
	\cardcontent{Du darfst dir jederzeit (auch bevor du diese Karte ausspielst) den Müllstapel ansehen. Wenn du den \emph{SCHURKEN} ausspielst und sich mindestens 1 Karte mit Kosten von \coin[3] bis \coin[6] im Müll befindet, musst du diese nehmen. Zeige sie deinen Mitspielern und lege sie auf deinen Ablagestapel. Nur wenn dies nicht der Fall ist, muss jeder Mitspieler die obersten 2 Karten seines Nachziehstapels aufdecken und eine Karte mit Kosten von \coin[3] bis \coin[6] entsorgen (nach seiner Wahl). Karten mit \hex (aus \emph{Empires}) bzw. \potion (aus \emph{Alchemisten}) in den Kosten kosten nie \coin[3] bis \coin[6].}
\end{tikzpicture}
\hspace{-0.6cm}
\begin{tikzpicture}
	\card
	\cardstrip
	\cardbanner{banner/white.png}
	\cardicon{icons/coin.png}
	\cardprice{5}
	\cardtitle{Vogelfreie}
	\cardcontent{Wähle eine offenliegende Aktionskarte aus dem Vorrat, die weniger kostet als diese Karte (\emph{VOGELFREIE}) und behandle diese Karte (\emph{VOGELFREIE}) so, als wäre sie die gewählte Aktionskarte. Normalerweise bedeutet dies, dass du die Anweisungen der Aktionskarte ausführst. Darüber hinaus nimmt \emph{VOGELFREIE} den Namen, die Kosten und die Kartentypen der gewählten Aktionskarte an – bis sie aus dem Spiel geht.

	\smallskip

	Karten von einem leeren Vorratsstapel, Karten von Nicht-Vorratsstapeln und verdeckt liegende Karten eines Vorratsstapels (z.B. noch nicht aufgedeckte RITTER) dürfen von \emph{VOGELFREIE} nicht imitiert werden.

	\smallskip

	Imitiert \emph{VOGELFREIE} eine Karte, die sich selbst irgendwo hinbewegt (z.B. abgelegt oder entsorgt wird), bewegt sich stattdessen \emph{VOGELFREIE} dorthin. Wird \emph{VOGELFREIE} als Dauerkarte gespielt, bleibt sie solange im Spiel, wie es auch die Dauerkarte selbst tun würde.

	\smallskip

	Solange \emph{VOGELFREIE} im Spiel ist, ist sie für alle Belange die imitierte Aktionskarte, nicht \emph{VOGELFREIE}. Zählt eine andere Karte zum Beispiel die Anzahl von Exemplaren einer bestimmten Karte und imitiert \emph{VOGELFREIE} jene Karte, zählt \emph{VOGELFREIE} als jene Karte.}
\end{tikzpicture}
\hspace{-0.6cm}
\begin{tikzpicture}
	\card
	\cardstrip
	\cardbanner{banner/white.png}
	\cardicon{icons/coin.png}
	\cardprice{6}
	\cardtitle{Altar}
	\cardcontent{Wenn du keine Karte entsorgen kannst, nimmst du trotzdem eine Karte vom Vorrat, die bis zu \coin[5] kostet. Karten mit \hex (aus \emph{Empires}) bzw. \potion (aus \emph{Alchemisten}) in den Kosten sowie Karten, die nicht zum Vorrat gehören (z.B. \emph{VERRÜCKTER}), dürfen nicht genommen werden.}
\end{tikzpicture}
\hspace{-0.6cm}
\begin{tikzpicture}
	\card
	\cardstrip
	\cardbanner{banner/white.png}
	\cardicon{icons/coin.png}
	\cardprice{6}
	\cardtitle{Jagdgründe}
	\cardcontent{Wenn du diese Karte entsorgst und dich entscheidest, drei \emph{ANWESEN} zu nehmen, und es sind nicht genügend im Vorrat, nimm so viele, wie du kannst. Diese Karte selbst gibt dir nicht das Recht, sich selbst zu entsorgen. Du kannst sie nur entsorgen, wenn das durch einen anderen Effekt im Spiel (z.B. die Anweisung auf einer anderen Karte) ausgelöst wird.}
\end{tikzpicture}
\hspace{-0.6cm}
\begin{tikzpicture}
	\card
	\cardstrip
	\cardbanner{banner/gold.png}
	\cardicon{icons/coin.png}
	\cardprice{0*}
	\cardtitle{Beute}
	\cardcontent{Diese Karte gehört nicht zum Vorrat – sie kann nur auf Grund der Anweisung auf den Königreichkarten \emph{BANDITENLAGER}, \emph{MARODEUR} und/oder \emph{RAUBZUG} genommen werden. Wenn du diese Karte ausspielst, lege sie auf ihren Stapel zurück.

	\smallskip

	Allgemeiner Hinweis: Anweisungen, eine Karte zu nehmen, ohne explizite Angabe woher, beziehen sich nicht auf \emph{BEUTE}, \emph{SÖLDNER} und/oder \emph{VERRÜCKTER}, sondern nur auf Karten aus dem Vorrat.}
\end{tikzpicture}
\hspace{-0.6cm}
\begin{tikzpicture}
	\card
	\cardstrip
	\cardbanner{banner/white.png}
	\cardicon{icons/coin.png}
	\cardprice{0*}
	\cardtitle{Söldner}
	\cardcontent{Diese Karte gehört nicht zum Vorrat – sie kann nur auf Grund der Anweisung auf der Königreichkarte \emph{GASSENJUNGE} genommen werden.

	\smallskip

	2 deiner Handkarten zu entsorgen ist optional. Wenn du mindestens 2 Handkarten hast und dich dafür entscheidest, musst du 2 Karten entsorgen. Du kannst dich nicht entscheiden, nur 1 Karte zu entsorgen. Hast du nur 1 Handkarte, kannst du diese entsorgen. Aber nur wenn du genau 2 Handkarten entsorgst, erhältst du + 2 Karten und +\coin[2] und die Mitspieler müssen ihre Handkarten reduzieren. Spieler, die bereits 3 oder weniger Handkarten haben, müssen keine Karten ablegen.

	\smallskip

	Mitspieler, die auf das Ausspielen dieser Angriffskarte mit einer Reaktionskarte reagieren wollen, müssen dies tun, bevor du dich entscheidest, Handkarten zu entsorgen oder nicht. Entsorgst du Karten, die eine \enquote{Wenn du diese Karte entsorgst}-Anweisung enthalten, entsorge erst beide Karten und entscheide dich dann, in welcher Reihenfolge du die Anweisungen ausführst.

	\smallskip

	Allgemeiner Hinweis: Anweisungen, eine Karte zu nehmen, ohne explizite Angabe woher, beziehen sich nicht auf \emph{BEUTE}, \emph{SÖLDNER} und/oder \emph{VERRÜCKTER}, sondern nur auf Karten aus dem Vorrat.}
\end{tikzpicture}
\hspace{-0.6cm}
\begin{tikzpicture}
	\card
	\cardstrip
	\cardbanner{banner/white.png}
	\cardicon{icons/coin.png}
	\cardprice{0*}
	\cardtitle{Verrückter}
	\cardcontent{Diese Karte gehört nicht zum Vorrat – sie kann nur auf Grund der Anweisung auf der Königreichkarte \emph{EREMIT} genommen werden.

	\smallskip

	Wenn du diese Karte ausspielst, lege sie auf ihren Stapel zurück, wenn du das kannst (dies ist nicht optional). Nur wenn du das tust, erhältst du für jede deiner Handkarten + 1 Karte.
	
	\smallskip
	
	Allgemeiner Hinweis: Anweisungen, eine Karte zu nehmen, ohne explizite Angabe woher, beziehen sich nicht auf \emph{BEUTE}, \emph{SÖLDNER} und/oder \emph{VERRÜCKTER}, sondern nur auf Karten aus dem Vorrat.}
\end{tikzpicture}
\hspace{-0.6cm}
\begin{tikzpicture}
	\card
	\cardstrip
	\cardbanner{banner/red.png}
	\cardicon{icons/coin.png}
	\cardprice{1}
	\cardtitle{\footnotesize{Unterschlupf}}
	\cardcontent{Es gibt drei verschiedene UNTERSCHLUPF-Karten, die jeweils 6x im Spiel enthalten sind. Sie haben keinen Stapel (weder im Vorrat, noch außerhalb des Vorrats) und können niemals gekauft werden.

	\medskip

	\emph{Hütte}: Diese Karte ist eine Reaktionskarte und darf von der Hand eingesetzt werden, wenn du eine beliebige Punktekarte kaufst. Du darfst die HÜTTE dann entsorgen, erhältst aber nichts dafür, du bist sie nur los.

	\smallskip

	\emph{Totenstadt}: Wenn du diese Karte ausspielst, erhältst du + 2 Aktionen. 
	
	\smallskip
	
	\emph{Verfallenes Anwesen}: Diese Karte ist eine Punktekarte – ist aber 0 \victorypoint wert.
	Wenn du diese Karte entsorgst, ziehe 1 Karte nach. Diese Karte selbst gibt dir nicht das Recht, sich selbst zu entsorgen. Du kannst sie nur entsorgen, wenn das durch einen anderen Effekt im Spiel (z.B. die Anweisung auf einer anderen Karte) ausgelöst wird.}
\end{tikzpicture}
\hspace{-0.6cm}
\begin{tikzpicture}
	\card
	\cardstrip
	\cardbanner{banner/brown.png}
	\cardicon{icons/coin.png}
	\cardprice{0}
	\cardtitle{Ruinen}
	\cardcontent{Jede der 5 RUINEN ist jeweils 10x im Spiel enthalten. Der RUINEN-Stapel ist Teil des Vorrats, gehört aber nicht zu den 10 Königreichkarten des Spiels. Der RUINEN-Stapel ist ein gemischter Stapel. RUINEN sind auch Aktionskarten und werden entsprechend als solche behandelt.}
\end{tikzpicture}
\hspace{-0.6cm}
\begin{tikzpicture}
	\card
	\cardstrip
	\cardbanner{banner/white.png}
	\cardtitle{\scriptsize{Empfohlene 10er Sätze\qquad}}
	\cardcontent{\emph{Dark Ages:}
	
	\smallskip

	\emph{Leichenzug:} \\ 
	Festung, Jagdgründe, Katakomben, Kultist, Marktplatz, Mundraub, Prozession, Ritter, Vogelfreie, Waffenkammer

	\smallskip 
	
	\emph{Spiel mit dem Teufel} \\ 
	Banditenlager, Grabräuber, Lagerraum, Landstreicher, Lumpensammler, Medium, Ratten, Raubzug, Schrotthändler, Weiser

	\smallskip 
	
	\emph{Dark Ages und \textit{Basisspiel}:} 

	\smallskip 
	
	\emph{Auf und Ab:} \\ 
	Armenhaus, Barde, Eremit, Jagdgründe, Medium, \textit{Geldverleiher}, \textit{Hexe}, \textit{Keller}, \textit{Thronsaal}, \textit{Werkstatt}

	\smallskip 
		
	\emph{Dark Ages und \textit{Empires}:} 

	\smallskip

	\emph{Grab des Rattenkönigs:} \\ 
	Festung, Lagerraum, Leichenkarren, Ratten, Raubzug, \textit{Aufstieg (Ereignis)}, \textit{Grabmal (Landmarke)} - \textit{Legionär}, \textit{Opfer}, \textit{Schlösser}, \textit{Stadtviertel}, \textit{Wagenrennen}}
\end{tikzpicture}
\hspace{-0.6cm}
\begin{tikzpicture}
	\card
	\cardstrip
	\cardbanner{banner/white.png}
	\cardtitle{\scriptsize{Spielvorbereitung (1/2)}\qquad}
	\cardcontent{Sortiert die Karten und steckt sie entsprechend der Reihenfolge auf der \emph{Sortierhilfe} in den Schachteleinsatz.

	\smallskip

	Zum Spielen benötigt ihr das \emph{DOMINION Basisspiel} oder das \emph{Basiskarten-Set} (empfehlenswert mit mindestens einer weiteren Erweiterung). Legt alle Basiskarten sowie die \emph{FLÜCHE} und die Müllkarte (bzw. das Müll-Tableau aus dem \emph{Basisspiel 2. Edition}) wie gewohnt als Teil des Vorrats in die Tischmitte.
	
	\smallskip

	Verwendet ihr im Spiel die Königreichkarte \emph{LEHEN}, legt ihr entsprechend der Spieleranzahl die folgende Anzahl \emph{LEHEN} in den Vorrat: bei 2 Spielern: 8 Karten und bei 3+ Spielern: 12 Karten.
	
	\smallskip

	Verwendet ihr im Spiel ausschließlich Königreichkarten aus \emph{Dark Ages}, erhält jeder Spieler statt 3 \emph{ANWESEN} je 1 \emph{HÜTTE}, 1 \emph{TOTENSTADT} sowie 1 \emph{VERFALLENES ANWESEN}. Wird mit mindestens 1 Königreichkarte aus \emph{Dark Ages} gespielt, einigt euch vor Spielbeginn, ob ihr mit den UNTERSCHLUPF-Karten oder den \emph{ANWESEN} starten wollt.
	
	\smallskip

	Verwendet ihr im Spiel die Königreichkarte \emph{RITTER}, mischt ihr die Karten mit dem Typ RITTER, legt den Stapel verdeckt in den Vorrat und deckt die oberste Karte auf.}
\end{tikzpicture}
\hspace{-0.6cm}
\begin{tikzpicture}
	\card
	\cardstrip
	\cardbanner{banner/white.png}
	\cardtitle{\scriptsize{Spielvorbereitung (2/2)}\qquad}
	\cardcontent{Verwendet ihr im Spiel mindestens eine der Königreichkarten mit dem Typ PLÜNDERN (\emph{KULTIST}, \emph{LEICHENKARREN} und/oder \emph{MARODEUR}), mischt ihr alle 50 RUINEN (braun) und legt folgende Anzahl als verdeckten Stapel in den Vorrat - die oberste Karte wird aufgedeckt.\\
	bei 2 Spielern: 10 Karten \\
	bei 3 Spielern: 20 Karten \\
	bei 4 Spielern: 30 Karten \\
	bei 5 Spielern: 40 Karten \\
	bei 6 Spielern: 50 Karten \\

	\smallskip

	Verwendet ihr im Spiel mindestens eine der Königreichkarten \emph{BANDITENLAGER}, \emph{MARODEUR} oder \emph{RAUBZUG}, legt ihr alle \emph{BEUTE}-Karten als Stapel neben dem Vorrat bereit (der \emph{BEUTE}-Stapel gehört \emph{NICHT} zum Vorrat).

	\smallskip

	Verwendet ihr im Spiel die Königreichkarte \emph{EREMIT}, legt ihr alle \emph{VERRÜCKTEN}-Karten als Stapel neben dem Vorrat bereit (der \emph{VERRÜCKTEN}-Stapel gehört NICHT zum Vorrat).

	\smallskip

	Verwendet ihr im Spiel die Königreichkarte \emph{GASSENJUNGE}, legt ihr alle \emph{SÖLDNER}-Karten als Stapel neben dem Vorrat bereit (der \emph{SÖLDNER}-Stapel gehört \emph{NICHT} zum Vorrat).}
\end{tikzpicture}
\hspace{-0.6cm}
\begin{tikzpicture}
	\card
	\cardstrip
	\cardbanner{banner/white.png}
	\cardtitle{\scriptsize{Neue Regeln (1/7)}\qquad}
	\cardcontent{\tiny{\begin{Spacing}{1}
	\vspace{1em}
	\emph{Es gelten die Basisspielregeln mit folgenden Änderungen bzw. Ergänzungen:}

	\medskip

	\emph{UNTERSCHLÜPFE}: In \emph{Dark Ages} gibt es drei verschiedene UNTERSCHLUPF-Karten, die jeweils 6x im Spiel enthalten sind. Beim Spiel ausschließlich mit Karten aus \emph{Dark Ages} müssen, im Spiel mit Karten aus \emph{Dark Ages} und einer weiteren Edition oder Erweiterung können diese eingesetzt werden.

	\smallskip

	In der Spielvorbereitung erhält jeder Spieler 1 \emph{HÜTTE}, 1 \emph{TOTENSTADT} sowie 1 \emph{VERFALLENES ANWESEN} anstelle der üblichen 3 \emph{ANWESEN}. Die restlichen UNTERSCHLUPF-Karten werden in die Schachtel zurückgelegt. In den Vorrat werden wie üblich die \emph{ANWESEN} gelegt.

	\smallskip

	UNTERSCHLUPF-Karten haben keinen Stapel (weder im Vorrat, noch außerhalb des Vorrats) und können niemals gekauft oder genommen werden. Der \emph{BOTSCHAFTER} (aus \emph{Seaside}) darf keinen UNTERSCHLUPF zurücklegen.
	\end{Spacing}}}
\end{tikzpicture}
\hspace{-0.6cm}
\begin{tikzpicture}
	\card
	\cardstrip
	\cardbanner{banner/white.png}
	\cardtitle{\scriptsize{Neue Regeln (2/7)}\qquad}
	\cardcontent{\tiny{\begin{Spacing}{1}
	\vspace{1em}
	\emph{RUINEN}: Wird mindestens eine der Königreichkarten mit dem Kartentyp PLÜNDERN (\emph{KULTIST}, \emph{LEICHENKARREN} und/oder \emph{MARODEUR}) im Spiel verwendet, werden zusätzlich die braunen RUINEN-Karten benötigt. Jede der 5 RUINEN ist jeweils 10x im Spiel enthalten. Alle Karten werden in einem Stapel zusammengemischt – er enthält also 5 unterschiedliche Karten in zufälliger Reihenfolge.

	\smallskip

	Legt folgende Anzahl RUINEN verdeckt in den Vorrat:\\
	bei 2 Spielern: 10 Karten \\
	bei 3 Spielern: 20 Karten \\
	bei 4 Spielern: 30 Karten \\
	bei 5 Spielern: 40 Karten \\
	bei 6 Spielern: 50 Karten \\

	\smallskip

	Der RUINEN-Stapel ist Teil des Vorrats, gehört aber nicht zu den 10 Königreichkarten des Spiels. Wird er im Laufe des Spiels aufgebraucht, zählt er als leerer Stapel für das Eintreten der Spielende-Bedingung.

	\smallskip

	Die oberste Karte wird aufgedeckt. Die restlichen RUINEN werden in die Schachtel zurückgelegt. Der RUINEN-Stapel ist ein \emph{gemischter Stapel}. Karten daraus können nach den entsprechenden Regeln gekauft und genommen werden. Sie sind auch Aktionskarten und werden entsprechend als solche behandelt.
	\end{Spacing}}}
\end{tikzpicture}
\hspace{-0.6cm}
\begin{tikzpicture}
	\card
	\cardstrip
	\cardbanner{banner/white.png}
	\cardtitle{\scriptsize{Neue Regeln (3/7)}\qquad}
	\cardcontent{\tiny{\begin{Spacing}{1}
	\vspace{1em}
	\emph{RITTER}: In \emph{Dark Ages} gibt es 5 \emph{DAMEN} und 5 \emph{SIRS}, die jeweils nur 1x im Spiel enthalten sind und alle dem Kartentyp RITTER angehören. Werden die RITTER im Spiel verwendet, werden alle 10 Karten gemischt und der Stapel verdeckt in den Vorrat gelegt. Der RITTER-Stapel ist Teil des Vorrats und gehört zu den 10 Königreichkarten des Spiels. Bis auf die Ausnahme, dass auf diesen Stapel kein Marker für die \emph{HANDELSROUTE} (aus \emph{Blütezeit}) gelegt wird, auch wenn eine Punktekarte (\emph{DAME JOSEPHINE}) oben liegt, gelten alle Regelungen für Königreichkarten.

	\smallskip

	Die oberste Karte wird aufgedeckt. Der RITTER-Stapel ist ein \emph{gemischter Stapel}.
	\end{Spacing}}}
\end{tikzpicture}
\hspace{-0.6cm}
\begin{tikzpicture}
	\card
	\cardstrip
	\cardbanner{banner/white.png}
	\cardtitle{\scriptsize{Neue Regeln (4/7)}\qquad}
	\cardcontent{\tiny{\begin{Spacing}{1}
	\vspace{1em}
	\emph{Gemischte Stapel}: RITTER und RUINEN sind Kartenstapel, die aus unterschiedlichen Karten bestehen, gemischt und verdeckt in den Vorrat gelegt werden. Die oberste Karte wird aufgedeckt. Es liegt immer nur die oberste Karte offen. Erst wenn die oberste Karte vom Stapel genommen wird, wird die nächste Karte aufgedeckt. Gemischte Stapel dürfen nicht durchgesehen werden.

	\smallskip

	Wird eine Karte auf einen gemischten Stapel zurückgelegt (z.B. durch den \emph{BOTSCHAFTER} aus \emph{Seaside}), wird die zuvor offen ausliegende Karte wieder umgedreht.

	\smallskip

	Die einzelnen Karten gelten als Karten mit unterschiedlichem Namen, auch wenn sie vom gleichen Stapel stammen. Deckt ein Spieler zum Beispiel durch den \emph{TRIBUT} (aus \emph{Intrige 1. Edition}) zwei unterschiedliche RUINEN auf, erhält er + 4 Aktionen.

	\smallskip

	Muss ein Spieler auf Grund einer Anweisung eine Karte nennen, muss er eine konkrete Karte nennen, er darf zum Beispiel nicht RUINE nennen, sondern \emph{ZERSTÖRTES DORF} oder \emph{ÜBERLEBENDE}.

	\smallskip

	Anweisungen (oder Marker, z.B. Embargomarker aus \emph{Seaside}), die einen Stapel betreffen, betreffen alle Karten in einem \emph{gemischten Stapel}.

	\smallskip

	Anweisungen, die konkrete Karten betreffen, dürfen nur ausgeführt werden, wenn dies regulär möglich ist. So darf ein Spieler, der mit Hilfe des \emph{TALISMANS} einen \emph{ZERSTÖRTEN MARKT} kauft, nur einen weiteren vom Stapel nehmen, wenn ein weiterer \emph{ZERSTÖRTER MARKT} dort obenauf liegt.
	\end{Spacing}}}
\end{tikzpicture}
\hspace{-0.6cm}
\begin{tikzpicture}
	\card
	\cardstrip
	\cardbanner{banner/white.png}
	\cardtitle{\scriptsize{Neue Regeln (5/7)}\qquad}
	\cardcontent{\tiny{\begin{Spacing}{1}
	\vspace{1em}
	\emph{Andere Karten}: Die Karten \emph{BEUTE}, \emph{VERRÜCKTER} und \emph{SÖLDNER} sind jeweils mit einem kleinen * versehen und haben zwar Funktionen wie klassische Aktions- oder Geldkarten, gehören aber NICHT zum Vorrat und werden nur im Spiel verwendet, wenn dies durch die Verwendung bestimmter Königreichkarten notwendig wird:\\
	\emph{BEUTE}: Dieser Stapel wird nur verwendet, wenn mindestens eine der folgenden Karten im Spiel verwendet wird: \emph{BANDITENLAGER}, \emph{MARODEUR}, \emph{RAUBZUG}.\\
	\emph{VERRÜCKTER}: Dieser Stapel wird nur verwendet, wenn der \emph{EREMIT} im Spiel verwendet wird.\\
	\emph{SÖLDNER}: Dieser Stapel wird nur verwendet, wenn der \emph{GASSENJUNGE} im Spiel verwendet wird.

	\smallskip

	Der (oder die) jeweilige(n) Stapel wird/werden neben dem Vorrat bereit gelegt. Die Karten aus den Stapeln dürfen nur durch Anweisungen auf den entsprechenden Königreichkarten genommen werden. Auf andere Art und Weise dürfen diese Karten nicht von ihren Stapeln genommen werden. Insbesondere können sie nicht gekauft werden. Die Stapel werden für das Eintreten der Spielende-Bedingungen nicht beachtet.

	\smallskip

	Der \emph{BOTSCHAFTER} (aus \emph{Seaside}) darf keine der Karten auf den entsprechenden Stapel zurücklegen.
	\end{Spacing}}}
\end{tikzpicture}
\hspace{-0.6cm}
\begin{tikzpicture}
	\card
	\cardstrip
	\cardbanner{banner/white.png}
	\cardtitle{\scriptsize{Neue Regeln (6/7)}\qquad}
	\cardcontent{\tiny{\begin{Spacing}{1}
	\vspace{1em}
	\emph{Den Anschluss verlieren}: In seltenen Fällen kann es vorkommen, dass sich eine Karte, die bewegt (z.B. entsorgt) werden soll, nicht dort befindet, wo sie ein Effekt erwartet. Dies kann geschehen, wenn die Karte bereits vorher von einem anderen Effekt bewegt oder (auf dem Ablagestapel) verdeckt wurde. In diesem Fall kann der Effekt, der den Anschluss an die Karte verloren hat, die Karte nicht bewegen. Alle sonstigen Anweisungen auf der Karte des Effekts werden aber trotzdem ausgeführt.

	\smallskip

	\emph{Beispiel}:\\
	Wenn mit einer \emph{PROZESSION} ein \emph{VERRÜCKTER} gespielt wird, bekommt der Spieler zuerst +2 Aktionen, legt den \emph{VERRÜCKTEN} auf den \emph{VERRÜCKTEN}-Stapel zurück und zieht so viele Karten, wie er Handkarten hat. Anschließend werden die Anweisungen auf dem \emph{VERRÜCKTEN} erneut ausgeführt. Der Spieler bekommt also erneut +2 Aktionen, kann dann aber den \emph{VERRÜCKTEN} nicht erneut zurücklegen, weil er sich nicht im Spiel befindet. Weil auf dem \emph{VERRÜCKTEN} steht \enquote{Wenn du das tust:}, zieht der Spieler also auch nicht nochmal Karten. Anschließend möchte die \emph{PROZESSION} den \emph{VERRÜCKTEN} aus dem Spiel entsorgen, aber das klappt nicht, weil er nicht mehr dort ist. Ganz am Schluss bekommt der Spieler eine Aktionskarte, die \coin[1] mehr kostet als der \emph{VERRÜCKTE}, falls es eine solche im Vorrat gibt.
	\end{Spacing}}}
\end{tikzpicture}
\hspace{-0.6cm}
\begin{tikzpicture}
	\card
	\cardstrip
	\cardbanner{banner/white.png}
	\cardtitle{\scriptsize{Neue Regeln (7/7)}\qquad}
	\cardcontent{\tiny{\begin{Spacing}{1}
	\vspace{1em}
	\emph{Nicht-Vorratsstapel}: \emph{BEUTE}, \emph{VERRÜCKTER} und \emph{SÖLDNER} haben zwar eigene Stapel, gehören aber nicht zum Vorrat und dürfen nur genommen werden, wenn eine Anweisung auf einer Königreichkarte dies anweist. Darfst du laut Anweisung \enquote{eine Karte nehmen}, darfst du keine der o.g. Karten nehmen, da diese Anweisung sich immer auf eine Karte vom Vorrat bezieht und sich \emph{BEUTE}, \emph{VERRÜCKTER} und \emph{SÖLDNER} nicht im Vorrat befinden.

	\smallskip

	Die UNTERSCHLÜPFE \emph{HÜTTE}, \emph{TOTENSTADT} und \emph{VERFALLENES ANWESEN} gehören ebenfalls nicht zum Vorrat, haben aber keinen eigenen Stapel außerhalb des Vorrats.

	\medskip

	\emph{Geldkarten mit zusätzlichen Anweisungen}: Die Karten \emph{FALSCHGELD} und \emph{BEUTE} sind Geldkarten und werden in der entsprechenden Spielphase eingesetzt. Beim Ausspielen muss darauf geachtet werden, dass die zusätzlichen Anweisungen direkt beim Ausspielen der entsprechenden Geldkarte ausgeführt werden. Deshalb ist es wichtig, dass Geldkarten hintereinander ausgespielt werden.

	\medskip

	\emph{Mehrere Dinge passieren zur gleichen Zeit}: Passieren einem Spieler gleichzeitig mehrere Dinge, darf er selbst entscheiden, in welcher Reihenfolge sie eintreten.
	\end{Spacing}}}
\end{tikzpicture}
\hspace{-0.6cm}
\begin{tikzpicture}
	\card
	\cardstrip
	\cardbanner{banner/white.png}
	\cardtitle{\scriptsize{Neue Anweisungen (1/2)}\qquad}
	\cardcontent{\tiny{\begin{Spacing}{1}
	\vspace{1em}
	\enquote{\emph{Wenn du diese Karte entsorgst\dots}}: Viele Karten in \emph{Dark Ages} enthalten die Anweisung etwas zu tun, wenn diese Karte entsorgt wird. Die entsprechende Anweisung wird sofort ausgeführt, wenn die Karte entsorgt wird – das kann im Zug eines anderen Mitspielers sein oder auch zwischen den Anweisungen auf einer Aktionskarte. Entsorgst du gleichzeitig mehrere Karten mit \enquote{Wenn du diese Karte entsorgst\dots}, darfst du die Reihenfolge selbst bestimmen, in der du die entsprechenden Anweisungen ausführst. Die Karte mit \enquote{Wenn du diese Karte entsorgst\dots} gibt dir nicht selbst das Recht die Karte zu entsorgen – du benötigst eine Karte mit einer entsprechenden Anweisung, um die Karte zu entsorgen.

	\medskip

	\enquote{\emph{Im Spiel}}: Karten, die ein Spieler offen in seinem Spielbereich vor sich liegen hat, befinden sich im Spiel, bis sie abgelegt werden. Nicht im Spiel befinden sich zur Seite gelegte und entsorgte Karten, sowie Handkarten, Karten im Vorrat und in den Zieh- und Ablagestapeln. Auch Reaktionskarten, die als Reaktion aufgedeckt werden, befinden sich nicht im Spiel.

	\medskip

	\enquote{\emph{Ansehen}}: Der Spieler nimmt die angegebene(n) Karte(n), sieht sie sich an und legt sie – sofern nicht anders auf der ausgespielten Karte angewiesen – dorthin zurück, wo er sie her hat. Er darf die Karte(n) keinem Mitspieler zeigen.
	\end{Spacing}}}
\end{tikzpicture}
\hspace{-0.6cm}
\begin{tikzpicture}
	\card
	\cardstrip
	\cardbanner{banner/white.png}
	\cardtitle{\scriptsize{Neue Anweisungen (1/2)}\qquad}
	\cardcontent{\tiny{\begin{Spacing}{1}
	\vspace{1em}
	\enquote{\emph{Wähle eins}}: Der Spieler muss genau eine der aufgelisteten Anweisungen auswählen und sie – soweit möglich – ausführen. Dabei darf er auch eine Anweisung wählen, die er nicht oder nur teilweise ausführen kann. Die restlichen Anweisungen haben für diesen Zug keine Wirkung. Wird die Karte später im Spiel wieder ausgespielt, darf der Spieler natürlich eine andere Wahl treffen.

	\medskip

	\emph{Karten mit \enquote{unterschiedlichen Namen}}: Ist die Rede von Karten mit unterschiedlichen Namen (NICHT Kartentyp), ist immer der Name oben auf den jeweiligen Karten relevant, auch wenn die Karten von demselben Stapel stammen (z.B. bei gemischten Stapeln).

	\medskip

	\emph{Eine Karte \enquote{nennen}}: Muss ein Spieler eine Karte nennen, sagt er den Namen einer Karte (nicht eines Kartenstapels); das ist wichtig zum Beispiel bei gemischten Stapeln.

	\medskip

	\enquote{\emph{Diese Karte}}: Enthält eine Karte eine Anweisung, die sich auf \enquote{diese Karte} bezieht, ist IMMER die Karte gemeint, auf der die Anweisung steht, niemals eine andere (jene) Karte, auf die innerhalb der Anweisung Bezug genommen wird.
	\end{Spacing}}}
\end{tikzpicture}
\hspace{0.6cm}
